\documentclass [aspectratio=169]{beamer}
\beamertemplatenavigationsymbolsempty
\usetheme{Boadilla}
\usepackage{textpos}
\usepackage{graphicx}
\usepackage{float}
\usepackage{hyperref}
\usepackage{caption}
\usepackage{subcaption}
\usepackage{algorithm,algpseudocode}
\usepackage[export]{adjustbox}
\usepackage{tikz}
\usepackage[square,numbers]{natbib}
\usepackage[byname]{smartref}
\usetikzlibrary{positioning}
\usetikzlibrary{arrows, shapes, decorations, automata, backgrounds, fit, petri, calc}

\newcommand*{\logofont}{\fontfamily{phv}\selectfont}

\definecolor{uwopurple}{RGB}{0, 140, 58}

\title[]{Weekly Presentation (18.08.25 - 28.08.25)}
\author[]{Yuxin XUE}
\institute[]{Faculty IV -- Electrical Engineering and Computer Science\\
Technische Universität Berlin}
\date{\today}

% Math-like helpers (kept from template)
\newtheorem{thm}{Theorem}[section]
\newtheorem{lem}[thm]{Lemma}
\newtheorem{defn}[thm]{Definition}
\newtheorem{eg}[thm]{Example}
\newtheorem{ex}[thm]{Exercise}
\newtheorem{conj}[thm]{Conjecture}
\newtheorem{cor}[thm]{Corollary}
\newtheorem{claim}[thm]{Claim}
\newtheorem{rmk}[thm]{Remark}
\newcommand{\ie}{\emph{i.e.} }

% Colors
\setbeamercolor{title in head/foot}{bg=white}
\setbeamercolor{author in head/foot}{bg=white}
\setbeamercolor{date in head/foot}{fg=uwopurple}
\setbeamercolor{date in head/foot}{bg=white}
\setbeamercolor{title}{fg=uwopurple}
\setbeamerfont{title}{series=\bfseries}
\setbeamercolor{frametitle}{fg=uwopurple}
\setbeamerfont{frametitle}{series=\bfseries}
\setbeamercolor{block title}{bg=uwopurple!30,fg=black}
\setbeamercolor{item}{fg=uwopurple}
\setbeamercolor{caption name}{fg=uwopurple!70!}

% Logo on non-title pages
\logo{\includegraphics[height=0.7cm]{qenergy.png}\vspace*{-.035\paperheight}\hspace*{.83\paperwidth}}

\begin{document}

{
\setbeamertemplate{logo}{}
\begin{frame}
    \titlepage
    \begin{textblock*}{4cm}(0.5cm,-7cm)
        \includegraphics[width=4cm]{qenergy.png}
    \end{textblock*}
    \begin{textblock*}{8cm}(5.0cm,-7.0cm)
        \huge \color{uwopurple}{$\Bigr\rvert$ \hspace{0.15cm} \textbf{Qenergy Platform}}
    \end{textblock*}
\end{frame}
}

% --- Context -----------------------------------------------------
\begin{frame}{Context \& Objectives}
\begin{itemize}
    \item \textbf{Current Goal}: Build and validate the weekly report analysis pipeline (prototype)
    \begin{itemize}
        \item \textit{Upload} unstructured reports (\texttt{.docx}) $\rightarrow$ \textit{Parse} per-project updates
        \item \textit{Persist} to DB (project history) $\rightarrow$ \textit{Analyze} \& \textit{Show} on frontend
    \end{itemize}
    \item Scope this sprint: \textbf{Phase 2A} (Data foundation), \textbf{2B} (Report upload), \textbf{2B-} (Importer schema), initial plan for \textbf{2B+} (LLM parsing).
    \item Non-goals (this sprint): full analysis UI, WS streaming, long-run performance tuning.
\end{itemize}
\end{frame}

% --- Architecture snapshot ---------------------------------------
\begin{frame}{System Snapshot (Backend–Frontend–DB)}
\begin{itemize}
    \item \textbf{Backend}: FastAPI, SQLAlchemy, Alembic; endpoints for health, upload, bulk-import.
    \item API: prefix \texttt{/api}, CORS for \texttt{http://localhost:3000}.
    \item \textbf{DB}: PostgreSQL (canonical schema: \texttt{projects}, \texttt{project\_history}, \texttt{report\_uploads}, \texttt{weekly\_report\_analysis}).
    \item \textbf{Frontend}: Service layer decoupled from UI; upload popup + bulk folder picker.
    \item Env wiring: \texttt{NEXT\_PUBLIC\_API\_URL} $\rightarrow$ backend; Azure \texttt{AZURE\_OPENAI\_*} planned.
    \item \textbf{LLM (planned)}: LangChain + Azure OpenAI (env: \texttt{AZURE\_OPENAI\_*}).
\end{itemize}
\end{frame}

% --- Progress last two weeks ------------------------------------
\begin{frame}{Progress (Last Two Weeks)}
\textbf{Environment Setup}
\begin{itemize}
    \item FastAPI scaffold with CORS; DB connectivity; Alembic migrations.
    \item Frontend service layer created; mock $\rightarrow$ real API ready.
    \item Cross-platform scripts: install / start / test / demo.
    \item Conda env \texttt{qenergy-backend}; dev server \texttt{pnpm dev:fe}; ports: 8002/3000/5432.
    \item Env keys: \texttt{DATABASE\_URL}, \texttt{NEXT\_PUBLIC\_API\_URL}, Azure \texttt{AZURE\_OPENAI\_*}.
\end{itemize}
\vspace{0.4em}
\textbf{System Design \& Timeline}
\begin{itemize}
    \item Phases 2A–2E defined with estimates; prototype window identified.
    \item Canonical schema fixed; API types aligned.
\end{itemize}
\end{frame}

% --- Phase 2A ----------------------------------------------------
\begin{frame}{Phase 2A — Data Foundation \& Migrations \ (\checkmark)}
\begin{itemize}
    \item DB infrastructure online; \texttt{projects}, \texttt{project\_history}, \texttt{weekly\_report\_analysis}.
    \item Constraints \& indexes: \texttt{project\_code} UNIQUE; category CHECK; read indexes for history.
    \item Alembic baseline + first migration applied cleanly.
\end{itemize}
\end{frame}

% --- Phase 2B ----------------------------------------------------
\begin{frame}{Phase 2B — Report Upload (Backend–Frontend) \ (\checkmark)}
\begin{itemize}
    \item \textbf{Upload popup} (single/bulk): parses docx paragraphs \& tables; preview rows on UI.
    \item \textbf{Bulk folder import} (frontend): filter filenames matching \texttt{YYYY\_CW\#\#\_\{DEV|EPC|FINANCE|INVESTMENT\}.docx}.
    \item Backend returns structured rows (no DB write yet when using popup); error handling standardized.
\end{itemize}
\end{frame}

% --- Phase 2B- ---------------------------------------------------
\begin{frame}{Phase 2B- — Report Importer (Schema \& Linking)}
\begin{itemize}
    \item \texttt{report\_uploads} created; \texttt{project\_history.source\_upload\_id} FK added.
    \item CLI importer supports single file \& folder; file-level \texttt{sha256} de-dup.
    \item Extended schema: \texttt{project\_history.source\_text} (TEXT, nullable) added for future LLM context.
    \item Acceptance tests (unit+integration) passing: de-dup, state transitions, bulk folder import.
    \item \textbf{Pending}: upload status writeback (\texttt{parsed/failed}) with notes; backref query cases; frontend$\leftrightarrow$backend importer check; README flow diagram.
\end{itemize}
\end{frame}

% --- Phase 2B+ plan ---------------------------------------------
\begin{frame}{Phase 2B+ — LLM Parsing (Plan)}
\begin{itemize}
    \item \textbf{Model wiring}: LangChain + Azure OpenAI (\texttt{AZURE\_OPENAI\_API\_KEY}, \texttt{ENDPOINT}, \texttt{DEPLOYMENT}, \texttt{API\_VERSION}).
    \item \textbf{Output}: \textit{list of rows} (one docx $\rightarrow$ multiple project histories) with Pydantic validation.
    \item \textbf{Prompting}: system (domain + JSON contract), user (chunk + filename meta).
    \item \textbf{Aggregation}: group bullets by project; merge into single concise summary per project/file.
    \item \textbf{Mapping}: \texttt{project\_name} $\rightarrow$ \texttt{project\_code}; set \texttt{entry\_type='Report'}; compute ISO Monday \texttt{log\_date}.
    \item \textbf{Status}: \textcolor{red}{not completed yet}.
\end{itemize}
\end{frame}

% --- What is done vs not ----------------------------------------
\begin{frame}{What’s Done vs. Not Yet}
\begin{columns}[t,onlytextwidth]
\column{0.48\textwidth}
\textbf{Done}
\begin{itemize}
    \item Backend \& DB foundation ready
    \item Frontend service layer + upload UI
    \item Folder upload + docx parsing (non-LLM)
    \item Importer schema \& FK linking
    \item Uploads de-dup + \texttt{received} status
    \item \texttt{source\_text} column added
\end{itemize}

\column{0.48\textwidth}
\textbf{Not Yet}
\begin{itemize}
    \item LLM parsing to multi-row JSON
    \item Upload status updates \& notes on errors
    \item Mapping table (\texttt{project\_name} $\rightarrow$ \texttt{project\_code}) seeding
    \item Frontend$\leftrightarrow$backend importer comms check; backref queries; docs diagram
\end{itemize}
\end{columns}
\end{frame}

% --- Challenges --------------------------------------------------
\begin{frame}{Current Challenges}
\begin{itemize}
    \item \textbf{Cross-end communication}: existing mock code sometimes diverges from the current design/functionality, complicating backend–frontend integration and coordination.
    \item \textbf{Provenance \& traceability}: link uploads$\to$history (\texttt{source\_upload\_id}), persist \texttt{source\_text}, archive path, backref queries.
    \item \textbf{Schema evolution}: add \texttt{report\_uploads}, \texttt{source\_text}, index on \texttt{source\_upload\_id}; consider \texttt{import\_runs} and row-level content hash.
    \item \textbf{LLM multi-row extraction}: strict JSON contract (Pydantic), chunking strategy, retries/backoff, validation with rule-based fallback.
    \item \textbf{Name mapping}: normalize variants; use \texttt{project\_code\_candidate} when available; unresolved handling pipeline.
    \item \textbf{Idempotency}: file \texttt{sha256} + row content hash to skip unchanged writes.
    \item \textbf{Date logic}: ISO-week Monday \texttt{log\_date} across year boundaries.
    \item \textbf{Preview vs persist}: keep \texttt{/reports/upload/bulk} preview-only; add importer endpoints, \texttt{DRY\_RUN} guard; TDD-first.
    \item \textbf{Observability}: run logging, per-file status (\texttt{received/parsed/failed}), error notes, metrics.
    \item \textbf{Performance \& cost}: token budgeting, chunk sizes, rate limiting (429), batching.
    \item \textbf{Security \& storage}: archive location, retention policy, access control, PII considerations.
\end{itemize}
\end{frame}

% --- Next two weeks ---------------------------------------------
\begin{frame}{Next Week (Plan)}
\begin{itemize}
    \item Finish 2B- acceptance: upload status \texttt{parsed/failed}, backref queries, README flow diagram.
    \item Implement 2B+ chain: LangChain \& Azure wiring, prompt, Pydantic schema, aggregation, mapping.
    \item Implement 2C: Weekly Reports Editor (cards) with save-to-DB (UPSERT on \texttt{project\_code, log\_date}).
    \item Prepare minimal analysis hooks for 2D (risk/similarity placeholders).
    \item Add ISO Monday date unit tests (year-boundaries) \& seed mapping source-of-truth.
    \item E2E guard for LLM: \texttt{AZURE\_OPENAI\_E2E=1}; retries/backoff for 429.
\end{itemize}
\end{frame}

% --- Timeline check ---------------------------------------------
\begin{frame}{Timeline Check}
\begin{itemize}
    \item Week ending \textbf{2025-08-29}: \textcolor{green}{2A + 2B completed} (as planned).
    \item Week ending \textbf{2025-09-05}: target \textcolor{blue}{2B+ minimal} + \textcolor{blue}{2D start}.
    \item \textbf{2B-}: mostly implemented; pending status writeback \& backref query cases.
    \item Prototype viability window: Week 3–5, depending on 2D scope.
\end{itemize}
\end{frame}

% --- Risks & mitigations ----------------------------------------
\begin{frame}{Risks \& Mitigations}
\begin{itemize}
    \item \textbf{Unstable LLM output} $\rightarrow$ Strict schema + retries; fallback to rule-based parsing if needed.
    \item \textbf{Mapping coverage} $\rightarrow$ Seed mapping table; log unresolved for manual review.
    \item \textbf{Schedule creep} $\rightarrow$ Keep UI minimal; prioritize importer + analysis core.
    \item \textbf{Ops issues} $\rightarrow$ One-click scripts; pin deps; add CI later.
\end{itemize}
\end{frame}

% % --- Appendix: Keys ---------------------------------------------
% \begin{frame}{Appendix — LLM Environment Keys (Azure)}
% \begin{itemize}
%     \item \texttt{AZURE\_OPENAI\_API\_KEY} — access key
%     \item \texttt{AZURE\_OPENAI\_ENDPOINT} — e.g., \texttt{https://qenergy-llm-openai.openai.azure.com}
%     \item \texttt{AZURE\_OPENAI\_DEPLOYMENT} — \emph{deployment name} you created (used as \texttt{model})
%     \item \texttt{AZURE\_OPENAI\_API\_VERSION} — e.g., \texttt{2024-10-21}
% \end{itemize}
% \end{frame}

% --- Closing -----------------------------------------------------
\begin{frame}{Summary}
\begin{itemize}
    \item Backend–Frontend–DB foundation is in place; uploads and schema linking work.
    \item Report Importer nearly there; LLM parsing is the next unlock.
    \item Clear two-week plan to reach prototype viability.
\end{itemize}
\end{frame}

\end{document}
